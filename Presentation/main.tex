
\documentclass{bredelebeamer}




%%%%%%%%%%%%%%%%%%%%%%%%%%%%%%%%%%%%%%%%%%%%%%%%



\title[Jeu de cartes en lignes]{Soutenance de projet : jeu de cartes en ligne}
% Titre du diaporama

\subtitle{TER de licence informatique, deuxième année}
% Sous-titre optionnel

\author{Maëlle Beuret, Othmane Farajallah, Bachar Rima}
% La commande \inst{...} Permet d'afficher l' affiliation de l'intervenant.
% Si il y a plusieurs intervenants: Marcel Dupont\inst{1}, Roger Durand\inst{2}
% Il suffit alors d'ajouter un autre institut sur le modèle ci-dessous.

\institute[UM]


\date{6 juin 2017}
% Optionnel. La date, généralement celle du jour de la conférence

\subject{Soutenance de projet TER L2}
% C'est utilisé dans les métadonnes du PDF



\logo{
\includegraphics[scale=0.04]{logo.jpg}
}



%%%%%%%%%%%%%%%%%%%%%%%%%%%%%%%%%%%%%%%%%%%%%%%%%%%%%%%%%%%%%%%%%%%%%
\begin{document}

\begin{frame}
  \titlepage
\end{frame}





\begin{frame}{Sommaire}
  \tableofcontents
  % possibilité d'ajouter l'option [pausesections]
\end{frame}




\section{Introduction}

\begin{frame}{Introduction}
Texte normal \alert{Texte Alert}  \exemple{Texte exemple} \emph{Texte emphase}

\begin{columns}

\begin{column}{0.5\textwidth}
\begin{block}{Bloc simple}
\begin{itemize}
\item Premier point
\end{itemize}
\end{block}

\begin{exampleblock}{Bloc exemple}
\begin{itemize}
\item Premier point
\end{itemize}
\end{exampleblock}

\begin{alertblock}{Bloc alert}
\begin{itemize}
\item Premier point
\end{itemize}
\end{alertblock}

\end{column}

\begin{column}{0.5\textwidth}
\boiteviolette{
Une boite violette
}

\boiteorange{
Une boite orange
}

\boitegrise{
Une boite grise
}



\begin{tcolorbox}[tabvert,tabularx={X||Y|Y|Y|Y||Y}, boxrule=0.5pt, title=Mon tableau des prix]
Couleur & Prix 1  & Prix 2  & Prix 3 \\\hline\hline
Rouge   & 10.00   & 20.00   &  30.00 \\\hline
Vert    & 20.00   & 30.00   &  40.00  \\\hline
Bleu    & 30.00   & 40.00   &  50.00 \\\hline\hline
Orange  & 60.00   & 90.00   & 120.00 
\end{tcolorbox}

\end{column}

\end{columns}
\end{frame}




\section{Outils utilisés}

\begin{frame}{Outils utilisés}

\begin{block}{Bloc simple}
\begin{itemize}
\item Premier point
\item Second point
\item Troisième point
\end{itemize}
\end{block}

\begin{exampleblock}{Bloc exemple}
\begin{itemize}
\item Premier point
\item Second point
\item Troisième point
\end{itemize}
\end{exampleblock}

\begin{alertblock}{Bloc alert}
\begin{itemize}
\item Premier point
\item Second point
\item Troisième point
\end{itemize}
\end{alertblock}
\end{frame}


\section{Fonctionnalités du jeu}

\begin{frame}{Fonctionnalités du jeu}

\begin{columns}

\begin{column}{0.5\textwidth}
\boitejaune{
Ceci est \\
une boite jaune
}

\boiteorange{
Ceci est \\
une boite orange
}

\boitemarron{
Ceci est \\
une boite marron
}
\end{column}

\begin{column}{0.5\textwidth}
\boiteviolette{
Ceci est \\
une boite violette
}

\boitebleue{
Ceci est \\
une boite bleue
}

\boitegrise{
Ceci est \\
une boite grise
}

\end{column}

\end{columns}


\end{frame}



\section{Conclusion}

\begin{frame}{Conclusion}

	\begin{itemize}
		\item premier élément de liste,
		\item deuxième élément de liste,
		\item troisième élément de liste.
	\end{itemize}
\end{frame} 


\end{document}

