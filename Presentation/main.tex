\documentclass{bredelebeamer}
%%%%%%%%%%%%%%%%%%%%%%%%%%%%%%%%%%%%%%%%%%%%%%%%

\title[Jeu de cartes en lignes]{Soutenance de projet : jeu de cartes en ligne}
% Titre du diaporama

\subtitle{TER de licence informatique, deuxième année}
% Sous-titre optionnel

\author{Maëlle Beuret, Othmane Farajallah, Bachar Rima}
% La commande \inst{...} Permet d'afficher l'affiliation de l'intervenant.
% Si il y a plusieurs intervenants: Marcel Dupont\inst{1}, Roger Durand\inst{2}
% Il suffit alors d'ajouter un autre institut sur le modèle ci-dessous.

\institute[UM]

\date{6 juin 2017}
% Optionnel. La date, généralement celle du jour de la conférence

\subject{Soutenance de projet TER L2}
% C'est utilisé dans les métadonnes du PDF

\logo{\includegraphics[scale=0.15]{umLogo.png}}
%%%%%%%%%%%%%%%%%%%%%%%%%%%%%%%%%%%%%%%%%%%%%%%%%%%%%%%%%%%%%%%%%%%%%

\begin{document}

  \begin{frame}
    \titlepage
  \end{frame}

  %=======================SOMMAIRE========================%
  \begin{frame}{Sommaire}
    \tableofcontents
    % possibilité d'ajouter l'option [pausesections]
  \end{frame}

  %=======================INTRODUCTION========================%
  \section{Introduction}
  \begin{frame}{Introduction I}
    \begin{block}{Contexte du jeu}
      \begin{description}
        \item [Les Voyageurs de Kaeraly :]{Jeu de cartes coopératif en ligne}
        \pause
        \item [Le but du jeu :]{Tuer le Loup Alpha}
        \pause
        \item [Les moyens : ]{
          \begin{itemize}
            \item {Traverser différentes zones (forêt, rivière, plaine)}
            \pause
            \item {S'équiper d'armure ou d'armes}
            \pause
            \item {Utiliser des potions}
            \pause
          \end{itemize}
        }
      \end{description}
    \end{block}
    
    \begin{figure}[h!]
    	\centering
    	\includegraphics[scale=0.2]{cards.png}
    	\caption{Aperçu des différentes cartes du jeu}
    \end{figure}
\end{frame}
\begin{frame}{Introduction II}
    \begin{block}{Inspirations et origines}
      \begin{itemize}
        \item {Jeux de rôle sur table, et jeux de société (\textit{Munchkin})}
        \pause
        \item {Collaboration avec des illustrateurs (Suisse, Roumanie, Pays Bas)}
        \pause
        \item {Projet TER : un défi de développement}
      \end{itemize}
    \end{block}
    
    \begin{figure}
    	\centering
    	\includegraphics[scale=0.15]{munchkin.jpg}
    	\hspace{0.5cm}
    	\includegraphics[scale=0.13]{dungeons-dragons}
    	\caption{Le jeu \textit{Munchkin} et des livres de Dungeons \& Dragons}
    \end{figure}
 \end{frame}

  %=======================OUTILS UTILISÉS========================%
  \section{Outils utilisés}
  \begin{frame}{Outils utilisés I}
    \begin{block}{Technologies}
      \begin{description}
        \item [JavaScript :]{
        Langage de programmation événementiel pour la dynamisation des pages web côté client
        }
        \pause
        \begin{center}
          \line(1,0){250}
        \end{center}
        \item [NodeJS :]{
          \begin{itemize}
            \item {Une plateforme JavaScript côté serveur basée sur le moteur de JavaScript de Google Chrome V8, façonnée selon le modèle non-bloquant et intégrant des fonctions de \textit{callback}}
            \pause
            \item {Développement d'applications en temps réel, surtout les applications web et réseaux, en utilisant, parmi d'autres, la technologie de WebSocket qui permet un échange bilatéral synchrone entre les clients et le serveur}
            \pause
            \item {multi-plateforme, asynchrone, évenementielle, très rapide, monothread, libre et open-source (Licence MIT)}
          \end{itemize}
        }
      \end{description}
    \end{block}
  \end{frame}

  \begin{frame}{Outils utilisés II}
    \begin{block}{Technologies}
      \begin{description}
        \item[jQuery :] {
          \begin{itemize}
            \item {Une bibliothèque JavaScript libre et multi-plateforme permettant la dynamisation des pages web}
            \pause
            \item {Manipulation du DOM (\textit{Document Object Model})}
            \pause
            \item {Gestion des événements}
            \pause
            \item {Intégration d'animations et des effets visuels, etc}
          \end{itemize}
        }
        \pause
        \begin{center}
          \line(1,0){250}
        \end{center}
        \item [D3 :]{
        Une bibliothèque JavaScript offrant une myriade de fonctionnalités et permettant un rendu graphique élégant en intégrant une multitude d'animations et la possibilité de création/manipulation d'objets SVG
        }
        \pause
        \item [MongoDB :]{
          Un SGBD (Système de Gestion de Base de Données) multiplateforme, noSQL, libre et open-source, permettant le stockage et la manipulation des objets de type BSON (JSON Binaire) où JSON est une notation utilisé comme moyen très efficace pour le stockage, le transfert, et la manipulation des données
        }
      \end{description}
    \end{block}
  \end{frame}

  \begin{frame}{Outils utilisés III}
    \begin{block}{Outils de développement}
      \begin{description}
        \item [Atom :]{
          Un éditeur de texte multi-plateforme, libre et open-source développé par GitHub
        }
        \pause
        \item [Git :]{
          Un gestionnaire de versions permettant de stocker les différentes versions de l'ensemble des fichiers d'un projet localement et sur un serveur appelé un "dépôt" (\textit{repository}) tel que \textit{GitHub}, pour faciliter la collaboration entre développeurs
        }
      \end{description}
    \end{block}
  \end{frame}


  % \begin{exampleblock}{Bloc exemple}
  % \begin{itemize}
  % \item Premier point
  % \item Second point
  % \item Troisième point
  % \end{itemize}
  % \end{exampleblock}
  %
  % \begin{alertblock}{Bloc alert}
  % \begin{itemize}
  % \item Premier point
  % \item Second point
  % \item Troisième point
  % \end{itemize}
  % \end{alertblock}

%Texte normal \alert{Texte Alert}  \exemple{Texte exemple} \emph{Texte emphase}

% \begin{columns}
%
% \begin{column}{0.5\textwidth}
% \begin{block}{Bloc simple}
% \begin{itemize}
% \item Premier point
% \end{itemize}
% \end{block}
%
% \begin{exampleblock}{Bloc exemple}
% \begin{itemize}
% \item Premier point
% \end{itemize}
% \end{exampleblock}
%
% \begin{alertblock}{Bloc alert}
% \begin{itemize}
% \item Premier point
% \end{itemize}
% \end{alertblock}
%
% \end{column}
%
% \begin{column}{0.5\textwidth}
% \boiteviolette{
% Une boite violette
% }
%
% \boiteorange{
% Une boite orange
% }
%
% \boitegrise{
% Une boite grise
% }
%
%
%
% \begin{tcolorbox}[tabvert,tabularx={X||Y|Y|Y|Y||Y}, boxrule=0.5pt, title=Mon tableau des prix]
% Couleur & Prix 1  & Prix 2  & Prix 3 \\\hline\hline
% Rouge   & 10.00   & 20.00   &  30.00 \\\hline
% Vert    & 20.00   & 30.00   &  40.00  \\\hline
% Bleu    & 30.00   & 40.00   &  50.00 \\\hline\hline
% Orange  & 60.00   & 90.00   & 120.00
% \end{tcolorbox}
%
% \end{column}
%
% \end{columns}


\section{Fonctionnalités du jeu}

\begin{frame}{Fonctionnalités du jeu}
 \begin{block}{Architecture du site}
 \pause
  Site web composé de trois pages différentes:
   \begin{itemize}
     \item Page d'accueil
     \item Règles du jeu
     \item Remerciements
   \end{itemize}
 \end{block}
\end{frame}

\begin{frame}{Fonctionnalités}
\begin{figure}[h!]
			\centering
			\includegraphics[scale=0.25]{manual.png}
			\label{fig:manual}
			\caption{Aperçu de la page du jeu et ses différents éléments.}
		\end{figure}
  \end{frame}
   \begin{frame}{Démonstration}
  \end{frame}
% \begin{frame}{lol}
% \begin{columns}
%
% \begin{column}{0.5\textwidth}
% \boitejaune{
% Ceci est \\
% une boite jaune
% }
%
% \boiteorange{
% Ceci est \\
% une boite orange
% }
%
% \boitemarron{
% Ceci est \\
% une boite marron
% }
% \end{column}
%
% \begin{column}{0.5\textwidth}
% \boiteviolette{
% Ceci est \\
% une boite violette
% }
%
% \boitebleue{
% Ceci est \\
% une boite bleue
% }
%
% \boitegrise{
% Ceci est \\
% une boite grise
% }
%
% \end{column}
%
% \end{columns}
% \end{frame}



\section{Conclusion}

\begin{frame}{Conclusion}
\begin{block}{Fonctionnalités intégrées}
    \begin{itemize}
      \item {Les mécanismes de base comme le tour par tour et le tirage de cartes}
      \pause
      \item {L’affichage graphique de tous les éléments}
      \pause
      \item {L’implémentation des éléments cartographiques}
      \pause
      \item {Les fonctions d’interaction avec les cartes}
    \end{itemize}
  \end{block}

	\begin{block}{Fonctionnalité manquante}
	    \begin{itemize}
	      \pause
	      \item {Le combat avec le monstre}
	    \end{itemize}
	  \end{block}
\end{frame}

\begin{frame}{Conclusion}
\begin{block}{Apports personnels}
\begin{itemize}
\item {Apprentissage d'un nouveau langage de programmation : \textbf{JS}}
\pause
\item {L'utilisation de technologies très puissantes telles que \textbf{NodeJS }et les \textbf{WebSockets}}
\pause
\item { Enrichissement de notre expérience en gestion de projets}
\end{itemize}
\end{block}

\begin{block}{Perspectives}
	\begin{itemize}
		\item{Ajout de système de classes}
		\pause
		\item{Possibilités de donner des cartes à un autre joueur}
		\pause
		\item{Ajout d'animations et effets audio}
		\pause
		\item{Ajout de nouvelles cartes}
	\end{itemize}
	
\end{block}
\end{frame}
\end{document}
